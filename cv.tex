%%%%%%%%%%%%%%%%%
% This is an sample CV template created using altacv.cls
% (v1.3, 10 May 2020) written by LianTze Lim (liantze@gmail.com). Now compiles with pdfLaTeX, XeLaTeX and LuaLaTeX.
%
%% It may be distributed and/or modified under the
%% conditions of the LaTeX Project Public License, either version 1.3
%% of this license or (at your option) any later version.
%% The latest version of this license is in
%%    http://www.latex-project.org/lppl.txt
%% and version 1.3 or later is part of all distributions of LaTeX
%% version 2003/12/01 or later.
%%%%%%%%%%%%%%%%

%% If you are using \orcid or academicons
%% icons, make sure you have the academicons
%% option here, and compile with XeLaTeX
%% or LuaLaTeX.
% \documentclass[10pt,a4paper,academicons]{altacv}

%% Use the "normalphoto" option if you want a normal photo instead of cropped to a circle
% \documentclass[10pt,a4paper,normalphoto]{altacv}

\documentclass[10pt,a4paper,ragged2e,withhyper]{altacv}
%% AltaCV uses the fontawesome5 and academicons fonts
%% and packages.
%% See http://texdoc.net/pkg/fontawesome5 and http://texdoc.net/pkg/academicons for full list of symbols. You MUST compile with XeLaTeX or LuaLaTeX if you want to use academicons.

% Change the page layout if you need to
\geometry{left=1.25cm,right=1.25cm,top=1.5cm,bottom=1.5cm,columnsep=1.2cm}

% The paracol package lets you typeset columns of text in parallel
\usepackage{paracol}

% Change the font if you want to, depending on whether
% you're using pdflatex or xelatex/lualatex
\ifxetexorluatex
  % If using xelatex or lualatex:
  \setmainfont{Roboto Slab}
  \setsansfont{Lato}
  \renewcommand{\familydefault}{\sfdefault}
\else
  % If using pdflatex:
  \usepackage[rm]{roboto}
  \usepackage[defaultsans]{lato}
  % \usepackage{sourcesanspro}
  \renewcommand{\familydefault}{\sfdefault}
\fi

% Change the colours if you want to
\definecolor{SlateGrey}{HTML}{2E2E2E}
\definecolor{LightGrey}{HTML}{666666}
\definecolor{DarkPastelRed}{HTML}{450808}
\definecolor{PastelRed}{HTML}{8F0D0D}
\definecolor{GoldenEarth}{HTML}{E7D192}
\colorlet{name}{black}
\colorlet{tagline}{PastelRed}
\colorlet{heading}{DarkPastelRed}
\colorlet{headingrule}{GoldenEarth}
\colorlet{subheading}{PastelRed}
\colorlet{accent}{PastelRed}
\colorlet{emphasis}{SlateGrey}
\colorlet{body}{LightGrey}

% Change some fonts, if necessary
\renewcommand{\namefont}{\Huge\rmfamily\bfseries}
\renewcommand{\personalinfofont}{\footnotesize}
\renewcommand{\cvsectionfont}{\LARGE\rmfamily\bfseries}
\renewcommand{\cvsubsectionfont}{\large\bfseries}


% Change the bullets for itemize and rating marker
% for \cvskill if you want to
\renewcommand{\itemmarker}{{\small\textbullet}}
\renewcommand{\ratingmarker}{\faCircle}

\begin{document}
\name{Lars Quentin}
\tagline{Informatikstudent, Programmierer}

\personalinfo{%
  \email{lars.quentin@stud.uni-goettingen.de}
  \location{G\"ottingen, Deutschland}
  \homepage{lquenti.de}
  \github{lquenti}
  \gitlabgwdg{lars.quentin}
  \linkedin{lars-quentin-68375217a}
  %% You MUST add the academicons option to \documentclass, then compile with LuaLaTeX or XeLaTeX, if you want to use \orcid or other academicons commands.
  % \orcid{0000-0000-0000-0000}
  %% You can add your own arbtrary detail with
  %% \printinfo{symbol}{detail}[optional hyperlink prefix]
  % \printinfo{\faPaw}{Hey ho!}[https://example.com/]
  %% Or you can declare your own field with
  %% \NewInfoFiled{fieldname}{symbol}[optional hyperlink prefix] and use it:
  % \NewInfoField{gitlab}{\faGitlab}[https://gitlab.com/]
  % \gitlab{your_id}
}

\makecvheader
%% Depending on your tastes, you may want to make fonts of itemize environments slightly smaller
% \AtBeginEnvironment{itemize}{\small}

%% Set the left/right column width ratio to 6:4.
\columnratio{0.6}

% Start a 2-column paracol. Both the left and right columns will automatically
% break across pages if things get too long.
\begin{paracol}{2}
\cvsection{Berufliche Laufbahn}

\cvevent{IT-Support und Webentwickler}{Max-Planck-Institut f\"ur biophysikalische Chemie}{2017 September -- heute}{G\"ottingen}
\cvtag{Python}
\cvtag{Django}
\cvtag{Linux}
\cvtag{Clientsupport}
\medskip
\begin{itemize}
\item T\"agliches L\"osen komplexer IT-Probleme
\item Entwurf und Implementation eines Webfrontends f\"ur\\Dateiarchivierung in Django
\end{itemize}

\divider

\cvevent{Full-Stack PHP-Entwickler}{Universit\"at des Dritten Lebensalters G\"ottingen e.V.}{2020 Februar -- 2021 Januar}{G\"ottingen}
\cvtag{PHP}
\cvtag{ES6+}
\cvtag{Bootstrap}
\cvtag{docker}
\medskip
\begin{itemize}
\item Wartung und Weiterentwicklung eines internen PHP-Frameworks
\end{itemize}

\divider

\cvevent{Tutor Programmierpraktikum}{Georg-August-Universit\"at G\"ottingen}{2019 April -- 2019 Oktober}{G\"ottingen}
\cvtag{Java}
\cvtag{Git}
\cvtag{Swing}
\cvtag{Graphentheorie}
\medskip
\begin{itemize}
\item Betreuung von Bachelorstudenten
\item Aufgabe: Videospiel in Java implementieren (rundenbasiert)
\item Durchschnittsnote meiner Gruppen: 1,4
\end{itemize}

\cvsection{Letzten Projekte}

\cvevent{\href{https://gitlab.gwdg.de/numerikgang/curvepy}{curvepy}, WIP}{\href{https://numerikgang.pages.gwdg.de/}{Numerikgang}}{}{}
\begin{itemize}
\item High Performance Pythonlibrary zur Berechnung von Bezierkurven
\item Hohe unit sowie property-based test coverage
\item Verschiedenste Parallelismusstrategien
\end{itemize}

\divider

\cvevent{Programmierkurs f\"ur Juristen}{eLegal e.V., \href{elegal-göttingen.de}{elegal-göttingen.de}}{}{}
\begin{itemize}
\item Eine Einf\"uhrung f\"ur an Legal-Tech interessierten Studenten
\item Zur Lehre entstanden: \href{https://github.com/elegal-ev/Keres}{Keres, eine Word-Templateengine mit GUI}
\item Wegen Corona dieses Jahr abgesagt
\end{itemize}

\medskip

% use ONLY \newpage if you want to force a page break for
% ONLY the current column
\newpage

%% Switch to the right column. This will now automatically move to the second
%% page if the content is too long.
\switchcolumn

\cvsection{Technologien}

\cvskill{Python}{5}
\divider
\cvskill{Linux}{5}
\divider
\cvskill{Java}{4}
\divider
\cvskill{PHP}{3}
\divider
\cvskill{Git}{3}
\divider
\cvskill{C}{3}

\cvsection{Ausbildung}
\cvevent{Angewandte Informatik B.Sc}{Georg-August-Universit\"at G\"ottingen}{2017 Oktober -- heute}{}
\divider
\cvevent{Fachhochschulreife Informatik}{BBS 2 G\"ottingen}{2015 -- 2017}{}

\cvsection{Interessen}
\cvtag{Serverapplikationen}\\
\cvtag{Algorithmen \& Datenstrukturen}\\
\cvtag{Programmiersprachen}
\cvtag{Typsicherheit}
\cvtag{Systementwicklung}
\cvtag{HPC}

\end{paracol}


\end{document}
